\documentclass[a4paper, 12pt]{article}
\usepackage[top=2cm, bottom=2cm, left=2.5cm, right=2.5cm]{geometry}
\usepackage[utf8]{inputenc}
\begin{document}

\begin{center}
\textbf{Equação polinomial do 2º grau.}
\end{center}

\begin{flushright}
\textit{Equação polinomial do 2º grau.}
\end{flushright}

\begin{flushleft}
\underline{Equação polinomial do 2º grau.}
\end{flushleft}

\begin{flushleft}
\textbf{\textit{\underline{Equação polinomial do 2º grau.}}}
\end{flushleft}

Equação polinomial do 2º grau. atualizado né!

Uma equação da forma: $$ax^2 + bx +c = 0$$ Com $a \neq 0$, será chamada de equação
polinomial do 2º grau. 

A solução dessa equação é dada por
$$x = \frac{-b \pm \sqrt{b^2 - 4ac}} {2a}$$
\end{document}